\documentclass[11pt]{article}

%% FONTS
%% To get the default sans serif font in latex, uncomment following line:
 \renewcommand*\familydefault{\sfdefault}
%%
%% to get Arial font as the sans serif font, uncomment following line:
%% \renewcommand{\sfdefault}{phv} % phv is the Arial font
%%
%% to get Helvetica font as the sans serif font, uncomment following line:
% \usepackage{helvet}
\usepackage[small,bf,up]{caption}
\renewcommand{\captionfont}{\footnotesize}
\usepackage[left=1in,right=1in,top=1in,bottom=1in]{geometry}
\usepackage{graphics,epsfig,graphicx,float,subfigure,color}
\usepackage{amsmath,amssymb,amsbsy,amsfonts,amsthm}
\usepackage{url}
\usepackage{boxedminipage}
\usepackage[sf,bf,tiny]{titlesec}
 \usepackage[plainpages=false, colorlinks=true,
   citecolor=blue, filecolor=blue, linkcolor=blue,
   urlcolor=blue]{hyperref}
\usepackage{enumitem}

\newcommand{\todo}[1]{\textcolor{red}{#1}}
% see documentation for titlesec package
% \titleformat{\section}{\large \sffamily \bfseries}
\titlelabel{\thetitle.\,\,\,}

\newcommand{\bs}{\boldsymbol}
\newcommand{\alert}[1]{\textcolor{red}{#1}}
\setlength{\emergencystretch}{20pt}

\begin{document}


\begin{center}
  \vspace*{-2cm}
{\small MATH-GA 2012.001 and CSCI-GA 2945.001, Georg Stadler \&
  Dhairya Malhotra (NYU Courant)}
\end{center}
\vspace*{.5cm}
\begin{center}
\large \textbf{%%
High Performance Computing \\
Anthony Trubiano \\
Assignment 3 (due April 1, 2019) }
\end{center}

% ****************************

\noindent {\bf Note:} All computations were performed with an Intel Core i7-8750H Processor with base frequency 2.20 GHz. The maximum main memory bandwidth is 41.8 GB/s. At 16 double 
precision operations per cycle, the theoretical max flop rate would be about 35.2 GFlops/s. It has $6$ cores and can reach $12$ threads through hyper-threading. 

\begin{enumerate}
% --------------------------

\item {\bf Approximating Special Functions Using Taylor Series and Vectorization.}  I improved the accuracy of the function sin4\_vec() simply by adding more terms to the Taylor series using the Vector class. Including terms up to $O(x^{11})$ gives the same error as the sin4\_taylor() code given but runs about 2.5 times faster. 


\item {\bf Parallel Scan in OpenMP} Here we parallelize the scan operation using $p$ threads. Let $n$ be the length of the vector. The parallel implementation takes in the number of threads, then assigns each a chunk of size $p/n$. If there is a remainder, the last threads gets assigned these entries. Each thread then performs the scan operation on its chunk. We then compute a correction in serial by adding the final entry of each chunk to all entries in the following chunk. The speed of the algorithm as a function of number of threads is shown in the following table for $n=100,000,000$. 

\begin{table}[h!] 
	\centering
	\begin{tabular}{c | c}
		Threads & Time (s)  \\
		\hline
		Serial & 0.224\\
		1 & 0.287\\ 
		2 & 0.188\\
		3 & 0.153\\
		4 & 0.141\\
		5 & 0.135\\
		6 & 0.132\\
	\end{tabular}
	\caption{Time taken (in seconds) for the scan operation on a vector of length $n=100,000,000$ as a function of thread number. }
\end{table}

From this table we see that using $2$ threads is enough to get slightly better performance while using $4$ threads approximately halves the amount of time taken. We don't see $p$ times speed up because the parallel implementation has to perform the fix calculations that the serial version does not, which takes a non-negligible amount of time. Additionally, if we make $n$ smaller, by a factor of $10$ for example, the parallel version is about the same if not slower than the serial version. We need a large enough vector to get noticeable improvements from parallelization.  














\end{enumerate}

\end{document}
