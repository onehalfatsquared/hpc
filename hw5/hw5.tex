\documentclass[11pt]{article}

%% FONTS
%% To get the default sans serif font in latex, uncomment following line:
 \renewcommand*\familydefault{\sfdefault}
%%
%% to get Arial font as the sans serif font, uncomment following line:
%% \renewcommand{\sfdefault}{phv} % phv is the Arial font
%%
%% to get Helvetica font as the sans serif font, uncomment following line:
% \usepackage{helvet}
\usepackage[small,bf,up]{caption}
\renewcommand{\captionfont}{\footnotesize}
\usepackage[left=1in,right=1in,top=1in,bottom=1in]{geometry}
\usepackage{graphics,epsfig,graphicx,float,subfigure,color}
\usepackage{amsmath,amssymb,amsbsy,amsfonts,amsthm}
\usepackage{url}
\usepackage{boxedminipage}
\usepackage[sf,bf,tiny]{titlesec}
 \usepackage[plainpages=false, colorlinks=true,
   citecolor=blue, filecolor=blue, linkcolor=blue,
   urlcolor=blue]{hyperref}
\usepackage{enumitem}

\newcommand{\todo}[1]{\textcolor{red}{#1}}
% see documentation for titlesec package
% \titleformat{\section}{\large \sffamily \bfseries}
\titlelabel{\thetitle.\,\,\,}

\newcommand{\bs}{\boldsymbol}
\newcommand{\alert}[1]{\textcolor{red}{#1}}
\setlength{\emergencystretch}{20pt}

\begin{document}


\begin{center}
  \vspace*{-2cm}
{\small MATH-GA 2012.001 and CSCI-GA 2945.001, Georg Stadler \&
  Dhairya Malhotra (NYU Courant)}
\end{center}
\vspace*{.5cm}
\begin{center}
\large \textbf{%%
High Performance Computing \\
Anthony Trubiano \\
Assignment 4 (due April 15, 2019) }
\end{center}

% ****************************

\noindent {\bf Note:} All CPU computations were performed with an Intel Core i7-8750H Processor with base frequency 2.20 GHz. The maximum main memory bandwidth is 41.8 GB/s. At 16 double 
precision operations per cycle, the theoretical max flop rate would be about 35.2 GFlops/s. It has $6$ cores and can reach $12$ threads through hyper-threading. 

\begin{enumerate}
% --------------------------

\item {\bf Matrix Vector Multiplication on a GPU} We perform a matrix vector multiplication both on a CPU and GPU and compare the bandwidth achieved on each. We compare CPU and GPU results to check for errors. We use a vector size of $N = 16384$. The CPU implementation uses OpenMP on my machine with $8$ threads.

\begin{table}[h!] 
	\centering
	\begin{tabular}{c | c}
		Machine & Bandwidth (GB/s)  \\
		\hline
		CPU & 5.5\\
		Cuda 1 & 25.8\\ 
		Cuda 2 & 237\\
		Cuda 3 & 0.14?\\
		Cuda 4 & 41.2\\
		Cuda 5 & 24.5\\
	\end{tabular}
	\caption{Bandwidth for a matrix-vector multiply using different GPUs, with a system size of $N=16384$. }
\end{table}


\item {\bf 2D Jacobi Method on a GPU} Here we apply Jacobi iteration to solve the 2D Laplace equation on the square with homogeneous Dirichlet boundary conditions both on a CPU and GPU. The residuals are computed from the CPU and GPU to see if we get the same result. We then compare the time each takes to perform $1000$ iterations of the method on a $1000\times 1000$ matrix. 

\begin{table}[h!] 
	\centering
	\begin{tabular}{c | c}
		Machine & Time (s)  \\
		\hline
		CPU & 2.57\\
		Cuda 1 & 0.2\\ 
		Cuda 2 & 0.1\\
		Cuda 3 & 0.85\\
		Cuda 4 & 0.16\\
		Cuda 5 & 0.24\\
	\end{tabular}
	\caption{Time taken (in seconds) for the Jacobi method for $1000$ iterations on a $1000\times 1000$ system. }
\end{table}

Looking at the results, we see in the best case (CUDA2), a speedup of almost 20 to 50 times the performance on a CPU using OpenMP with $8$ threads, which is quite significant. On average over the GPUs, we see a speedup of about $5$ to $8$ times the CPU performance. I am not sure why, but on CUDA3, my result was incorrect and the computation took longer than the CPU (matrix-vector multiply). This is strange considering my code works fine on the other systems. 



\item {\bf Project Proposal} I will be working with Ondrej Maxian and Tristan Goodwill broadly on "Parallel Algorithms for CFD". We will begin by writing a solver for Poisson's Equation on a periodic domain using a parallel FFT. We will then implement a parallel force spreading and grid interpolation scheme to use with the Immersed Boundary Method on a periodic domain. 














\end{enumerate}

\end{document}
